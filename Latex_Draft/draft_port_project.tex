\documentclass[12pt]{article}

\title{Coca-Cola Stock Forecasting using ARMA Model}

\author{Zach Joachim, Taylor Jones, Trevor Wai, Carson Watkin}

\begin{document}

\maketitle

\begin{abstract}
 The goal of this paper is to build a model to forcast stock prices and profits of the Coco-Cola Company. Using an Autoregressive Integrated Moving Average (ARIMA) model to create this forcast. Stock prices tend to follow trends during certain periods of time. We attempt to use this data to demonstrate that this data can be used to create reasonable predictions of the behavior of stock prices.\dots
 \end{abstract}

\section{Problem Statement and Motivation}

Stock forecasting is vital for a company that wants to be successfull and profitable. It helps companies 
with financial planning, allowing companies to allocate resources efficiently and make imformed decisions 
about investments and expenditures. It allows companies to identify and mitigate risks associated with a 
volitile market. Finally, along with many other things, forecasting provides a benchmark for the company's 
actual performance to be measured against. By comparing the forecasted stock prices with the actual prices, 
companies can evaluate the effectiveness of their strategies and operations.\par
Clearly the ability to forecast stock prices and other important quantities is of great use and interest to 
companies everywhere.

\section{Data}

The Coca-Cola stock price history data came from a dataset on Kaggle's website. The dataset contains 7 quantities 
indexed by the day of those quantities. These values are the stock's opening price, highesst price point in the day, 
lowest price point in the day, the closing price of the stock, the volume, dividends, and stock slpits of the 
stock for the day. Finding data like this was difficult to find for quantites like profits, margins, or quantities 
for specific drinks that fall under the Coca-Cola umbrella. However, the stock price for a company is a good indicator 
as to what the values are and where they are heading.\par
The stock price data reuqired a little cleaning. For about 50\% the data, the index or date for the stock info 
comes in the form of YYYY-MM-DD, while the other 50\% has the time next to the date, in the same format, 
YYYY-MM-DD HH:MM:SS-UTC\@. For the purposes of our analysis, we only needed the date, and the format it was in was acceptable, 
so what we needed to do was delete part of the index that gave the time. To do this, we stripped the index 
into its list of strings and only kept the part that gave the date, we made this into a new column called \texttt{df['date']}, 
and deleted the original column.

\section{Results}

\subsection{ARMIA}

\subsection{VARMAX}

\section{Analysis}

\section{Ethical Implications}

The primary ethical concern of our project is the application of our analysis. Individuals wanting to manipulate the stock market would be able to use our model as their evidence to artificially manipulate supply and demand. This interferes with the free and fair operation of the stock market and is illegal in the United States. An example of how manipulators can use our analysis to affect the market is through stock bashing. Our analysis could predict the fall of a stock price and then perpetrators can use this as evidence for stock bashing or convincing the public that a stock price is worth less. Therefore allowing them to purchase the stock at a cheaper price to try and later turn a profit. \par
Therefore we should add a disclaimer in our project that all of the analysis and predictions created are simply models and not prophecies of events to come. Therefore people understand that our analysis is simply trying to predict how the stock prices are going to fluxuate and this can simply be a factor in their buying and selling of shares.
\section{Conclusion}

\end{document}